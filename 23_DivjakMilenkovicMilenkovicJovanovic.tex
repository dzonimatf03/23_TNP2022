\documentclass{article}
\usepackage[utf8]{inputenc}
\usepackage{fancyhdr}
\usepackage{lipsum}
\usepackage{graphicx}
\usepackage{ulem}
\usepackage{wrapfig}

\title{BROJ $e$ I PRIMENA U FINANSIJAMA\\\large{Seminarski rad u okviru kursa}\\\large{Tehničko i naučno pisanje}\\ \large{Matematički fakultet}}

\author{}
\begin{document}

\maketitle

\begin{center}
    

\section*{\large{ANASTASIJA DIVJAK}}

\paragraph{\normalfont{Student 1.godine Matematičkog fakulteta Univerziteta u Beogradu, smer Informatika}}

\section*{\large{NIKOLINA MILENKOVIĆ}}

\paragraph{\normalfont Student 1.godine Matematičkog fakulteta Univerziteta u Beogradu, smer Informatika}

\section*{\large{KRISTINA MILENKOVIĆ}}

\paragraph{\normalfont Student 1.godine Matematičkog fakulteta Univerziteta u Beogradu, smer Informatika}

\section*{\large{NIKOLA JOVANOVIĆ}}


\paragraph{\normalfont {Student 1.godine Matematičkog fakulteta Univerziteta u Beogradu, smer Informatika}}



\paragraph{\\}
\end{center}
\paragraph{\textbf{REZIME: }\normalfont { Kada je otkriven 1618. godine, broju $e$ nije pridavan veliki značaj. Njegovu pravu ulogu otkriva znatno kasnije švajcarski matematičar Leonard Ojler. Ova konstanta iznosi $e$  2,71828... i predstavlja osnovu prirodnog logaritma. Veliku primenu ima i van matematike, a jedna od njegovih najbitnijih primena je u finansijama. Smatra se da je kao prvobitni problem broja $e$ upravo bio problem vezan za finansije. }}
\paragraph{\normalfont{\textbf{KLJUČNE REČI}: broj $e$, prirodan logaritam, finansije, Leonard Ojler}}




\newpage{}

\title{\Large\centering{{SADRŽAJ}}}

\section{\LARGE Rezultati istraživanja i diskusija}
    \subsection{\Large\normalfont Istorija broja $e$ \hspace{8.1cm} 3}
    \subsection{\Large\normalfont O Leonardu Ojleru \hspace{7.2cm} 3}
    \subsection{\Large\normalfont Osobina broja $e$ \hspace{7.88cm} 4}
    \subsection{\Large\normalfont Ojlerov identitet \hspace{7.7cm} 6}
    \subsection{\Large\normalfont Ojlerova kružnica \hspace{7.5cm} 7}
    \subsection{\Large\normalfont Primena broja $e$ u finansijama \hspace{4.85cm} 8}
    \subsection{\Large\normalfont O Bernuliju \hspace{8.7cm} 8}
    \subsection{\Large\normalfont Značaj u finansijama \hspace{6.75cm} 8}
    \subsection{\Large\normalfont Reference \hspace{9.1cm} 9} 



\newpage{}
\pagestyle{fancy}
\fancyhead{}
\fancyhead[RO,LE]{\textbf{Broj $e$ i primene u finansijama}}

\title{\Large\centering{{REZULTAT ISTRAŽIVANJA I DISKUSIJA}}}

\section*{\uline{Istorija broja $e$}}
\paragraph{\normalfont{Broj e se prvi put pojavljuje 1618.godine u logaritamskim tablicama tj. nakon Neperovog otkrića integrala. Tada mu nije pridavan veliki znacaj i njegovu ulogu u matematici i drugim oblastima otkriva znatno kasnije švajcarski matematičar i fizičar Leonard Ojler}}

\paragraph{\normalfont{Naime, u 17.veku švajcarski matematičar Danijel Bernuli ispitivao je kamatnu stopu i različite dohotke na osnovu učestalosti ulaganja. Ono što je zaključio, a što se sad smatra originalnim problemom broja $e$, jeste da se dobija bolji rezultat ako se češće ulaže novac 
i uzima kamata.}}

\paragraph{\normalfont{Pedesetak godina nakon ovoga, Ojler je napokon izračunao vrednost broja $e$ s obzirom da je Bernuli znao samo da se taj broj nalazi izmedju 2 i 3. Osim što ga je izračunao on je pronašao i formulu kojom je dokazao da je ovaj broj iracionalan.\\}}


\begin{wrapfigure}{r}{0.5\textwidth}
  \begin{center}
    \includegraphics[width=0.48\textwidth]{lojlermodified.jpg}
  \end{center}
  \caption{Leonardo Ojler}
\end{wrapfigure}

\section*{\uline{O Leonardu Ojleru}}

Leonard Ojler rodjen je u Bazelu 15.aprila 1707.godine.
Bio je jedan od najznačajnijih matematičara 18.veka.
Najviše je doprineo u matematičkoj notaciji jer je prvi počeo
da koristi $f(x)$ za zapis funkcije, moderan zapis trigonometrijskih funkcija,$e$ ,$\Sigma$
za sumu,$i$ kao imaginarnu jedinicu.
Pokazao je da se najkraće rastojanje izmedju dve tačke na zakrivljenoj površi pretvara u duž ukoliko se ta površ projektuje na ravan.
Medju manje poznatim Ojlerovim doprinosima nalazi se pokušaj formulisanja teorije muzike u potpunosti zasnovan na matematičkim idejama, koji je napravio napisavši 1739. godine Tentamen novae theoriae musicae,a zatim i brojna druga dela.
\newpage{}
\begin{document}
\pagestyle{fancy}
\fancyhead{}
\fancyhead[RO,LE]{\textbf{Broj $e$ i primene u finansijama}}
\section*{\uline{Primena broja e u finansijama}}
\begin{wrapfigure}{r}{0.5\textwidth}
  \begin{center}
    \includegraphics[width=0.33\textwidth]{bernuli.jpg}
  \end{center}
  \caption{Danijel Bernuli}
\end{wrapfigure}
\paragraph{\normalfont{Vrednost broja e u početku je računata u bankarse svrhe.
Kao što je rečeno, smatra se da je problem broja e bio vezan za finansije. Danijel Bernuli ispitivao je kamatnu stopu i različite dohotke na osnovu učestalosti ulaganja.}}
\section*{\uline{O Bernuliju}}
Sticao je znanja iz matematike i prirodnih nauka, predavao je matematiku, anatomiju, botaniku i fiziku. Bio je prijatelj Leonarda Ojlera, zajedno su saradjivali na više polja matematike i fizike. Različiti problemi koje je pokušavao da razreši (teorija elastičnosti, mehanika talasa) nagnali su ga da razvije takav matematički aparat kao što su diferencijlne jednačine i redovi.
\section*{\uline{Značaj u finansijama}}
Pretpostavimo da u banku ulažemo sumu novca h. Ako bi banka davala 100%-tnu kamatu,
za godinu dana mogli bismo da podignemo sumu novca 2h. Ukoliko bismo posle pola
godine podigli novac i opet ga uložili nakon godinu dana suma novca iznosila bi
\begin{equation} (1+\frac{1}{2})^2*x. \end{equation}
Ako bi smo ovaj postupak primenjivali svaki dan, nakon godinu dana suma novca
iznosila bi $((1+\frac{1}{365})^3^6^5*x)$. Sad, postavlja se pitanje koliko novca bi mogli da zaradimo kada bi banka računala složen interes n beskrajno malim vremenskim intervalima tzv. „neprekidno kapitalisanje“? Odgovor je:
\begin{equation} \lim_{n \rightarrow \infty} (1+\frac{1}{n})^n=e \end{equation}
\newpage{}
\section*{\uline {Zakljčak}}
\paragraph{\normalfont{U cilju približavanja teme čitaocu i lakšeg razumevanja, ovaj rad izrađen je kroz teorijska objašnjenja i praktične primere čija je tačnost potvrđena. Fokus ovog rada bio je na jednoj konstanti čija se važnost i primena ne ogledaju samo na polju matematike već i u svakodnevnom životu. Iako mi toga možda nismo svesni, ovo otkriće u matematici doprinelo je razvoju mnogih drugih sfera kao što su biologija, fizika, hemija, bankarstvo, računovodstvo... Pošto se broj e kao zasebna tema ne izučava toliko u školi već se spominje kroz druge pojave, nadam se da sam kroz ovaj rad uspela da objasnim značaj i istorijski razvitak ovog broja.}}

\section*{\uline {Literatura}}
\paragraph{\normalfont
{1. Vene T. Bogosavov, Zbirka rešenih zadataka iz matematike 3, za 3. razred srednje škole, Zavod za udzbenike i nastavna sredstva, Beograd, 2013.\\
2. S. Ognjanović, Ž. Ivanović, Krugova zbirka zadataka i testova iz matematike za 3. razred gimnazija i tehničkih škola, Krug, Beograd, 2010.\\
3.\url{ http://elementarium.cpn.rs/teme/sedam-najlepsih-brojeva/} \\
4.\url{https://www.matematika.edu.rs/saznajte-zanimljivosti-o-broju-e/}\\
5.\url{https://www.iserbia.rs/da-li-ste-znali/dan-broja-pi-broj-e-314-271/}\\
6.\url{http://poincare.matf.bg.ac.rs/}\\
7.\url{http://alas.matf.bg.ac.rs/~mn06070/Broj_e.pdf/}}}\\
\end{document}
